\documentclass[11pt]{book}		% drafthead style seems not to work w\book
\usepackage{graphicx}
\usepackage{amsmath}
\usepackage{color}
%\usepackage{fancyhdr}
%\pagestyle{fancy}
%\lhead{\today}
\setlength{\oddsidemargin}{0.50in}	% Note binding-side margin is wider,
\setlength{\evensidemargin}{0.00in}	% unlike Lamport's defaults
\setlength{\topmargin}{0.0in}
\setlength{\textheight}{8.0in}
\setlength{\textwidth}{6.0in}
\setlength{\parindent}{0.0in}
\setlength{\parskip}{0.5cm}



\title{CLINICAL NEUTRON THERAPY SYSTEM\\
	Control System Specification 1.0\\[1.0cm]}
%         Beam Diagnostics and Status Display \\[1.0cm]}
%        Includes Operator Interface Overview, Status Terminal, Operatons Terminal, Equipment Control Overview, Magnet System Control, Extraction System Control, Beam Diagnostics System Control, Safety System, Control System\\[1.0cm]}


\author{Robert Emery\\
	Ruedi Risler\\
	Dave Reid \\
	Mat Hicks \\
        Jonathan Jacky}
%	\\ 
%	Jonathan Unger \\
%	Stan Brossard \\ [0.5cm]
%	Radiation Oncology Department \\
%	University of Washington\\
%	Seattle, WA  98195 \\[0.5cm]
%	Technical Report 92-05-01

\date{\today}

\begin{document}

\chapter{SM23A System Control Testing} \label{ch:cyc-equip-ctl-beamline}

\vspace*{-.75in}
\today \\
\vspace*{.75in}
\\

This chapter describes in detail the functional requirements for the control of the CNTS Cyclotron Extraction System which includes the Electrostatic Deflector (Deflector), Extraction System mechanical positioning, Septum Temperature Monitoring System, Electro-magnetic Channel (EMC), and Internal Steering Magnet.  Controls for these devices will follow the standards laid out in section \ref{sect:cyc-equip-ctl-definitions} unless specified otherwise below.

\section{Steering Magnets X2A/Y2A/X3A/Y3A} \label{sect:cyc-equip-ctl-beamline-sm23a}

The Steering Magnet X2, X3, Y2, and Y1 Power Supplies (PSX2A, PSX3A, PSY2A, and PSY3A) each deliver up to 2.5A A and 40 V to produce magnetic fields used in the beamline transport system to center the beam first before quadropole 3A (X/Y2A), then finally before entering gantry (X/Y3A).  These power supplies operate in a current controlled mode with a minimum 12 bit I/O resolution, an output reproducibility of $\pm$.117 A and a longterm output stability of $\pm$ .02 mA over 24 hours.

\subsection{State Controls} \label{sect:cyc-equip-ctl-beamline-sm23a-state-controls}
(See section \ref{sect:cyc-equip-ctl-definitions} for control definitions not described below)

\begin{enumerate}
 
\item (ON,OFF) 

\color{red}
1) Press the On button in the operations console. Observe the following: 
	Does the PS come on? 
	Does SM23A:On:Status go from zero to 1 when you do this? 
	Does SM23A:Off:Status go from 1 to 0 when you do this? 
	Does SM23A:Condition:Initializing:Status get set to 1 then 0?
	Does SM23A:Reset:Interlocks:Write go to 1, then 0 again?

2) Set SM23A:X2/Y2/X3/Y3:CommError:Status to false, then turn on and see if this clears.

3) Set PSET to 1.0 A for X2. Turn the PS off, and see if the "SM23A:X2:Curr:Set" goes to zero before "SM23A:On:Status". While this occurs, SM23A:Condition:ShuttingDown:Status should go to 1 temporarily. Then, turn on the PS back on, and ensure that "SM23A:X2:Curr:Set" returns to 1.0 A.

4) Does pressing the Off button at the operations console turn off the PS? When you press off, does the "Shutting Down" light flash in both the operations and status display?

5) On the Operations Console, and Status display, observe that PSET says zero for all 4 coils, X2, Y2, X3, Y3 when the PS is off.

6) Place the device in local mode, and attempt to turn on the PS. You shouldn't be able to do this.

7) When you turn on the device locally, make sure SM23A:Condition:Initializing:Status gets set to 1 then 0.


\color{black}

 \item Initialize

\color{red}

Press the Initialize button in the operations console and observe the following: 
	Does the PV SM23A:Condition:Initializing:Status go from 0 to 1, then back to 0 during this process? 
	Ensure that it flashes "Initializing" in Yellow on the operations and status display screen. 
	Make sure you see the 4 PV's: SM23A:X2/X3/Y2/Y3:Init:Status go from 0 to 1 then back to 0.

Try setting Set SM23A:X2/Y2/X3/Y3:CommError:Status to false, then press Initialize and see if it goes back to true.
Turn the PV RF:HighPowerOn:Status from 0 to 1, and verify that you cannot press the Initialize button. 

I tried this: Then, run a caput SM23A:Init:String.PROC 1, and verify that SM23A:X2/X3/Y2/Y3:Init:Status stay at 0, and an Initization aborted b/c the beam is on message comes up on the CCC.

But then realized that the Input field of the calc record does not want to update when you do this. I am not sure why. I need to investigate this further. So i modify RF:HighPowerOn:Status with the adjust button on a probe, then the only way it will update on the calc record in SM23AInitialize.vdb, is if I restart the IOC. Otherwise, it won't read it. Maybe the record needs to have been processed? Does this only work for a CA link? Need to investigate this further.

\color{black}

 \item Reset

\color{red}

Press the Reset button on the operations console, and observe the PV SM23A:Reset:Interlocks:Write go to 1, then 0 again. Other then that, I don't know how to force the temperature interlocks to trigger.

\color{black}

\end{enumerate}

\subsubsection{Standby 1 to Standby 2 Transition} \label{sect:cyc-equip-ctl-beamline-sm23a-state-controls-sb1tosb2}

When the system is commanded to transition to the Standby 2 state as described in section \ref{sect:cyc-equip-ctl-controls-system-coordination-standby}, PSX2A, PSX3A, PSY2A, and PSY3A are turned on.

\color{red}

To Simulate going to SB2, you place a 1 on the field, BeamlineControl:GotoStandby2:BeamlineA.PROC. This should turn on sm23a. Verify that this happens.

Also, when line A and standby 2 are selected, sm23a should turn on. When any other line is selected in sb2, sm23a should turn off.

\color{black}

\subsubsection{Standby 2 to Standby 1 Transition} \label{sect:cyc-equip-ctl-beamline-sm23a-state-controls-sb2tosb1}

When the system is commanded to transition to the Standby 2 state as described in section \ref{sect:cyc-equip-ctl-controls-system-coordination-standby}, PSX2A, PSX3A, PSY2A, and PSY3A are turned off.

\color{red}

To Simulate going to SB1, you place a 1 on the field, BeamlineControl:GotoStandby1.PROC. This should turn off sm23a.  Verify that this happens.

\color{black}


\subsection{State Monitors} \label{sect:cyc-equip-ctl-beamline-sm23a-state-monitors}
(See section \ref{sect:cyc-equip-ctl-definitions} for state definitions not described below)

\begin{enumerate}

\item X2A/X3A/Y2A/Y3A (ON,OFF)

\color{red}
Already checked this in State Controls above.
\color{black}

\item X2A/X3A/Y2A/Y3A Local

\color{red}
Press the Local button the PS, and see if the "Local" buttons lights up in both the operations terminal and status terminal.
\color{black}

\item X2A/X3A/Y2A/Y3A Initializing

\color{red}
Already checked this in State Controls above.
\color{black}

 \item X2A/X3A/Y2A/Y3A Shutting Down

\color{red}
Already checked this in State Controls above.
\color{black}

\item +24 V Control Voltage Low

\color{red}
Not in control system.
\color{black}

\item X2A Circuit Resistance out of Tolerance

\color{red}
Not in control system.
\color{black} 

\item X3A Circuit Resistance out of Tolerance

\color{red}
Not in control system.
\color{black} 

\item Y2A Circuit Resistance out of Tolerance

\color{red}
Not in control system.
\color{black} 

\item Y3A Circuit Resistance out of Tolerance

\color{red}
Not in control system.
\color{black}

\end{enumerate}

\subsubsection{Device Interlocks}\label{sect:cyc-equip-ctl-beamline-sm23a-state-monitors-device-interlocks}

X2A/X3A/Y2A/Y3A Device Interlocks:
(Occurrence of interlock will turn off SM23A and not allow SM23A to be turned on unless otherwise noted)

\begin{enumerate}
 \item X2A/X3A/Y2A/Y3A Ground Fault - Reset Locally to Remove latch - No Remote Reset

\color{red}
In the modicon PLC, force read coil 10506 Off, and see if this is displayed in status display. Also "Device Interlock" should appear in red at above the operations console. I have not figured out a way to check to see if an actual ground fault would trigger this coil, or not. The reset command cannot fully be tested in this sense, eithier.
\color{black}

 \item X2A/Y2A Over Temperature - Reset to Remove latch

\color{red}
In the modicon PLC, force read coil 10510 On, and see if this is displayed in status display. Also "Device Interlock" should appear in red at above the operations console. I have not figured out a way to check to see if an actual ground fault would trigger this coil, or not. The reset command cannot fully be tested in this sense, eithier.
\color{black}

 \item X3A/Y3A Over Temperature - Reset to Remove latch

\color{red}
In the modicon PLC, force read coil 10511 On, and see if this is displayed in status display. Also "Device Interlock" should appear in red at above the operations console. I have not figured out a way to check to see if an actual ground fault would trigger this coil, or not. The reset command cannot fully be tested in this sense, eithier.
\color{black}

\end{enumerate}

\subsubsection{Particle Beam Interlocks}\label{sect:cyc-equip-ctl-beamline-sm23a-state-monitors-beam-interlocks}

Steering Magnet 2A/3A Particle Beam Interlocks:
(Occurrence of interlock will prevent RF system from attempting to accelerate a particle beam)

\begin{enumerate}
 \item X2A/X3A/Y2A/Y3A Initializing - Non-latching

\color{red}
Write SM23A:Condition:Initializing:Status from 0 to 1, and verify that SM23A:SubsystemOKSB2:Status goes from 1 to 0.
\color{black}


 \item X2A/X3A/Y2A/Y3A Off - Non-latching

\color{red}
I tried this: 

Write SM23A:Off:Status to 1, and verify that SM23A:SubsystemOKSB2:Status goes from 1 to 0.

But this didn't work b/c the input field of the calcout record will not update for a CA linked PV (outside of the IOC). Need to think of another way to test this.

I tried turning off the PS, and SM23A:SubsystemOKSB2:Status goes went from 1 to 0.
\color{black}


 \item X2A/X3A/Y2A/Y3A Shutting Down - Non-latching

\color{red}
Write SM23A:Condition:ShuttingDown:Status to 1, and verify that SM23A:SubsystemOKSB2:Status goes from 1 to 0.
\color{black}


 \item PSX2A Current $PREAD \geq PHIGH$ - Non-latching

\color{red}
With PSET=0.5, set PHIGH to 0, and verify that SM23A:SubsystemOKSB2:Status goes from 1 to 0. Verify the appropriate changes in the PREAD display in operations console and status display.
\color{black}

 \item PSX2A Current $PREAD \leq PLOW$ - Non-latching

\color{red}
With PSET=0.5, set PLOW to .6, and verify that SM23A:SubsystemOKSB2:Status goes from 1 to 0.Verify the appropriate changes in the PSET display in operations console and status display.
\color{black}

 \item PSX2A Current $\mid$PREAD-PSET$\mid$  $\geq$ PDiff - Non-latching

\color{red}
With PSET=0.5, set PDIFF to 0, and verify that SM23A:SubsystemOKSB2:Status goes from 1 to 0 (may have to wait). Verify the appropriate changes in the PSET display in operations console and status display. This is a hard one to get off b/c the PS's are so stable and precise. I had trouble getting the PDIFF to actually be read.
\color{black}

 \item PSX2A Communication Fault - Initialize to remove latch

\color{red}
Write SM23A:X2:CommError:Status from 0 to 1. Does SM23A:SubsystemOKSB2:Status goes from 1 to 0? Does a CommError Fushia colored light come on in the operations console and status display?
\color{black}

 \item PSY2A Current $PREAD \geq PHIGH$ - Non-latching

\color{red}
See X2A above.
\color{black}


 \item PSY2A Current $PREAD \leq PLOW$ - Non-latching

\color{red}
See X2A above.
\color{black}

 \item PSY2A Current $\mid$PREAD-PSET$\mid$  $\geq$ PDiff - Non-latching

\color{red}
See X2A above.
\color{black}

 \item PSY2A Communication Fault - Initialize to remove latch

\color{red}
See X2A above.
\color{black}

 \item PSX3A Current $PREAD \geq PHIGH$ - Non-latching

\color{red}
See X2A above.
\color{black}

 \item PSX3A Current $PREAD \leq PLOW$ - Non-latching

\color{red}
See X2A above.
\color{black}

 \item PSX3A Current $\mid$PREAD-PSET$\mid$  $\geq$ PDiff - Non-latching

\color{red}
See X2A above.
\color{black}

 \item PSX3A Communication Fault - Initialize to remove latch

\color{red}
See X2A above.
\color{black}

 \item PSY3A Current $PREAD \geq PHIGH$ - Non-latching

\color{red}
See X2A above.
\color{black}

 \item PSY3A Current $PREAD \leq PLOW$ - Non-latching

\color{red}
See X2A above.
\color{black}

 \item PSY3A Current $\mid$PREAD-PSET$\mid$  $\geq$ PDiff - Non-latching

\color{red}
See X2A above.
\color{black}

 \item PSY3A Communication Fault - Initialize to remove latch

\color{red}
See X2A above.
\color{black}

 \item X2A/X3A/Y2A/Y3A Watchdog - Initialize to remove latch

\color{red}
Write SM23A:OnandInit:Status from 1 to 0, which should stop the HB. Look for the PV SM23A:HeartbeatOK:Status to go from 1 to 0. You should also see SM23A:SubsystemOKSB2:Status goes from 1 to 0. You should see "SM23A WatchDog OK" appear in fushia on the status display screen. Then press the "Initialize" button, and observe that SM23A:HeartbeatOK:Status and SM23A:SubsystemOKSB2:Status return to 1, and "SM23A WatchDog OK" in the status display screen disappears.
\color{black}

\end{enumerate}


\subsection{Safety}\label{sect:cyc-equip-ctl-beamline-sm23a-safety}

Loss of control of the PSX2A, PSX3A, PSY2A or PSY3A will result in loss of control of the particle beam if one is being produced.  The primary protection against this is that the control system monitors the output current of the PSX2A, PSX3A, PSY2A and PSY3A.  If the output current is out of tolerance or in question, as described in section \ref{sect:cyc-equip-ctl-beamline-sm23a-state-monitors-beam-interlocks}, the control system will shut off the particle beam as described in section \ref{sect:cyc-equip-ctl-safety-sys-control-beam-control}.  The back up for this is provided by the particle beam hardwired safety interlock system (HSIS) (section \ref{sect:cyc-equip-ctl-safety-sys-hsis-beam}) that monitors the particle beam position and beam losses described in section \ref{ch:cyc-equip-ctl-beam-diagnostics}.


\color{red}

Bring up everything into SB2, and select beamline A in test. Make sure everything is okay, so that you can run beam down the A line onto target. Check that all the appropriate buttons are lit up, particularly the ``FC1 to Target'' button light should be on, and the FC1 ``Interlock'' light should be off. With the RF On, and FC1 Open (as well as FC2 and FC3), but NO CATHODE CURRENT, change PLOW below PSET and observe that the RF shuts off, and FC1 gets put in. Then observe the ``Interlock'' light is on next to FC1, and that the light on the ``FC1 to Tartget'' button turns off. Try opening FC1 and beam plug, and verify that they won't open. Turn the RF back on, with FC1 in. With the RF on, verify that you still cannot open FC1 or beamplug. Then change PLOW for SM23A back to the appropriate value and verify that the FC1 ``Interlock'' light goes away, the ``FC1 to Target'' button light turns on, the beam plug opens, and that now you can open the FC1 with the RF On.

Do this identical procedure for beamline A, in treat mode.

\color{black}


\subsection{Analog Control Parameters}\label{sect:cyc-equip-ctl-beamline-sm23a-analog-control}

\begin{enumerate}
 \item PSX2A Current PSET  n.nn A

\color{red}
Set a value for PSET, and verify that it appears on the PS, in operations console and status display, and on PREAD (in operations console and status display).
\color{black}

 \item PSX2A Current PLOW  n.nn A

\color{red}
Already tested in Particle Beam Interlock section.
\color{black}

 \item PSX2A Current PHIGH n.nn A

\color{red}
Already tested in Particle Beam Interlock section.
\color{black}


 \item PSX2A Current PDiff n.n A

\color{red}
Already tested in Particle Beam Interlock section.
\color{black}

 \item PSX2A Current PSEN  n

\color{red}
Change PSEN from 1 to 10, and verify that the tuning knob sensivity has changed in the tuning module. Here, you should also check that you can drag X2A into the tuning module, and control the PSET parameter with the knob. Move the knob, and observe the pset value on status display.
\color{black}

 \item PSY2A Current PSET  n.nn A

\color{red}
See X2A
\color{black}

 \item PSY2A Current PLOW  n.nn A

\color{red}
See X2A
\color{black}

 \item PSY2A Current PHIGH n.nn A

\color{red}
See X2A
\color{black}

 \item PSY2A Current PDiff n.n A

\color{red}
See X2A
\color{black}

 \item PSY2A Current PSEN  n

\color{red}
See X2A
\color{black}

 \item PSX3A Current PSET  n.nn A

\color{red}
See X2A
\color{black}

 \item PSX3A Current PLOW  n.nn A

\color{red}
See X2A
\color{black}

 \item PSX3A Current PHIGH n.nn A

\color{red}
See X2A
\color{black}

 \item PSX3A Current PDiff n.n A

\color{red}
See X2A
\color{black}

 \item PSX3A Current PSEN  n

\color{red}
See X2A
\color{black}

 \item PSY3A Current PSET  n.nn A

\color{red}
See X2A
\color{black}

 \item PSY3A Current PLOW  n.nn A

\color{red}
See X2A
\color{black}

 \item PSY3A Current PHIGH n.nn A

\color{red}
See X2A
\color{black}

 \item PSY3A Current PDiff n.n A

\color{red}
See X2A
\color{black}

 \item PSY3A Current PSEN  n

\color{red}
See X2A
\color{black}

\end{enumerate}


\subsection{Parameter Limits} \label{sect:cyc-equip-ctl-beamline-sm23a-analog-control-limits}

\begin{enumerate}
 \item PSX2A Current MinLimit = -1.3 A

\color{red}
This is hard encoded into the PS. If you do a software encode, it resets after it is turned off. The hardware encoded limit forces you to turn off the PS and turn it back on again if you go over. The operator is not notified of this. Should we keep it in?
\color{black}

 \item PSX2A Current MaxLimit = 1.3 A

\color{red}
This is hard encoded into the PS. If you do a software encode, it resets after it is turned off. The hardware encoded limit forces you to turn off the PS and turn it back on again if you go over. The operator is not notified of this. Should we keep it in?
\color{black}

 \item PSX2A Current PLOWWarn = PLOW + 0.5 A

\color{red}
With PSET=-0.51, and PLOW set to -1.0, verify the appropriate changes in the PSET display in operations console (both tuning modules) and status display.
\color{black}

 \item PSX2A Current PHIGHWarn = PHIGH - 0.5 A

\color{red}
With PSET=0.51, and PHIGH set to 1.0, verify the appropriate changes in the PSET display in operations console (both tuning modules) and status display.
\color{black}

 \item PSY2A Current MinLimit = -1.3 A

\color{red}
See X2A above.
\color{black}

 \item PSY2A Current MaxLimit = 1.3 A

\color{red}
See X2A above.
\color{black}

 \item PSY2A Current PLOWWarn = PLOW + 0.5 A

\color{red}
See X2A above.
\color{black}

 \item PSY2A Current PHIGHWarn = PHIGH - 0.5 A

\color{red}
See X2A above.
\color{black}

 \item PSX3A Current MinLimit = -1.3 A

\color{red}
See X2A above.
\color{black}

 \item PSX3A Current MaxLimit = 1.3 A

\color{red}
See X2A above.
\color{black}

 \item PSX3A Current PLOWWarn = PLOW + 0.5 A

\color{red}
See X2A above.
\color{black}

 \item PSX3A Current PHIGHWarn = PHIGH - 0.5 A

\color{red}
See X2A above.
\color{black}

 \item PSY3A Current MinLimit = -1.3 A

\color{red}
See X2A above.
\color{black}

 \item PSY3A Current MaxLimit = 1.3 A

\color{red}
See X2A above.
\color{black}

 \item PSY3A Current PLOWWarn = PLOW + 0.5 A

\color{red}
See X2A above.
\color{black}

 \item PSY3A Current PHIGHWarn = PHIGH - 0.5 A

\color{red}
See X2A above.
\color{black}

\end{enumerate}

\subsection{Analog Display Parameters} \label{sect:cyc-equip-ctl-beamline-sm23a-analog-display}

\begin{enumerate}
 \item PSX2A Current PSET  n.nn A

\color{red}
Tested in Analog Control Parameters.
\color{black}

 \item PSX2A Current PLOW  n.nn A

\color{red}
Tested in Analog Control Parameters.
\color{black}


 \item PSX2A Current PHIGH n.nn A

\color{red}
Tested in Analog Control Parameters.
\color{black}

 \item PSX2A Current PDiff n.nn A

\color{red}
Tested in Analog Control Parameters.
\color{black}

 \item PSX2A Current PSEN  n

\color{red}
Tested in Analog Control Parameters.
\color{black}

 \item PSX2A Voltage PREAD n.nn V

\color{red}
Not in control system.
\color{black}

 \item X2A Resistance PREAD n.n $\Omega$

\color{red}
Not in control system.
\color{black}

 \item X2A Resistance PLOW n.n $\Omega$

\color{red}
Not in control system.
\color{black}

 \item X2A Resistance PHIGH n.n $\Omega$

\color{red}
Not in control system.
\color{black}

 \item PSX3A Current PSET  n.nn A

\color{red}
See X2A above.
\color{black}


 \item PSX3A Current PLOW  n.nn A

\color{red}
See X2A above.
\color{black}

 \item PSX3A Current PHIGH n.nn A

\color{red}
See X2A above.
\color{black}

 \item PSX3A Current PDiff n.nn A

\color{red}
See X2A above.
\color{black}

 \item PSX3A Current PSEN  n

\color{red}
See X2A above.
\color{black}

 \item PSX3A Voltage PREAD n.nn V

\color{red}
Not in control system.
\color{black}

 \item X3A Resistance PREAD n.n $\Omega$

\color{red}
Not in control system.
\color{black}

 \item X3A Resistance PLOW n.n $\Omega$

\color{red}
Not in control system.
\color{black}

 \item X3A Resistance PHIGH n.n $\Omega$

\color{red}
Not in control system.
\color{black}

 \item PSY2A Current PSET  n.nn A

\color{red}
See X2A above.
\color{black}

 \item PSY2A Current PLOW  n.nn A

\color{red}
See X2A above.
\color{black}

 \item PSY2A Current PHIGH n.nn A

\color{red}
See X2A above.
\color{black}

 \item PSY2A Current PDiff n.nn A

\color{red}
See X2A above.
\color{black}

 \item PSY2A Current PSEN  n

\color{red}
See X2A above.
\color{black}

 \item PSY2A Voltage PREAD n.nn V

\color{red}
Not in control system.
\color{black}

 \item Y2A Resistance PREAD n.n $\Omega$

\color{red}
Not in control system.
\color{black}

 \item Y2A Resistance PLOW n.n $\Omega$

\color{red}
Not in control system.
\color{black}

 \item Y2A Resistance PHIGH n.n $\Omega$

\color{red}
Not in control system.
\color{black}

 \item PSY3A Current PSET  n.nn A

\color{red}
See X2A above.
\color{black}

 \item PSY3A Current PLOW  n.nn A

\color{red}
See X2A above.
\color{black}

 \item PSY3A Current PHIGH n.nn A

\color{red}
See X2A above.
\color{black}

 \item PSY3A Current PDiff n.nn A

\color{red}
See X2A above.
\color{black}

 \item PSY3A Current PSEN  n

\color{red}
See X2A above.
\color{black}

 \item PSY3A Voltage PREAD n.nn V

\color{red}
Not in control system.
\color{black}

 \item Y3A Resistance PREAD n.n $\Omega$

\color{red}
Not in control system.
\color{black}

 \item Y3A Resistance PLOW n.n $\Omega$

\color{red}
Not in control system.
\color{black}

 \item Y3A Resistance PHIGH n.n $\Omega$

\color{red}
Not in control system.
\color{black}

\end{enumerate}

\subsection{Implementation Notes} \label{sect:cyc-equip-ctl-beamline-sm23a-implementation}

The X2A, X3A, Y2A, and Y3A power supplies are commercial Kikusui DC supplies.  Digital control and monitoring of this power supply is via the MOD 1 PLC.  Analog control and monitoring are done by the control system using by way of Ethernet/GPIB via a Agilent E5810A GPIB gateway.





\chapter{SM23A Operations Console Testing}

\paragraph{Draggable Objects}

\begin{enumerate}
 \item ``X2A'' label. - Will assign Steering Magnet X2A output current parameter to Tuning Module or open Steering Magnet X2A Display in Device Operations region.

\color{red}
Already tested in Analog Control Parameters.
\color{black}

 \item ``X3A'' label. - Will assign Steering Magnet X3A output current parameter to Tuning Module or open Steering Magnet X3A Display in Device Operations region.

\color{red}
See X2A
\color{black}

 \item ``Y2A'' label. - Will assign Steering Magnet Y2A output current parameter to Tuning Module or open Steering Magnet Y2A Display in Device Operations region.

\color{red}
See X2A
\color{black}

 \item ``Y3A'' label. - Will assign Steering Magnet Y3A output current parameter to Tuning Module or open Steering Magnet Y3A Display in Device Operations region.

\color{red}
See X2A
\color{black}

\end{enumerate}

\paragraph{State Controls}

\begin{enumerate}
 \item SM23A (ON,OFF)

\color{red}
Already checked on State Controls of System Control Testing.
\color{black}

 \item SM23A Initialize (Button is disabled if device is off or if RF in High Power mode)

\color{red}
Already checked on State Controls of System Control Testing.
\color{black}

\end{enumerate}

\paragraph{State Indicators}

\begin{enumerate}
 \item SM23A (ON,OFF)

\color{red}
Already checked on State Controls of System Control Testing.
\color{black}

 \item SM23A Local

\color{red}
Already checked on State Monotors of System Control Testing.
\color{black}

 \item SM23A Initializing - Flashing Yellow

\color{red}
Already checked on State Controls of System Control Testing.
\color{black}

 \item SM23A Device Interlock - Red

\color{red}
Already checked on Device Interlocks Section of System Control Testing.
\color{black}

 \item SM23A Communication Fault - Fuchsia

\color{red}
Already checked on Particle Beam Interlocks of System Control Testing.
\color{black}

 \item SM23A Process Heartbeat - Fuchsia

\color{red}
Already checked on Particle Beam Interlocks of System Control Testing.
\color{black}

\end{enumerate}

\subsubsection{X2A Steering Magnet (X2A) Device Operations} \label{sect:cyc-op-interface-ops-terminal-device-ops-x2a}

\paragraph{Title} \label{sect:cyc-op-interface-ops-terminal-device-ops-x2a-title}

The title of the system will be X2A.  The color of the title bar will be teal.

\paragraph{State Controls}

\begin{enumerate}
\item None
\end{enumerate}

\paragraph{State Indicators}

\begin{enumerate}
 \item None
\end{enumerate}

\paragraph{Analog Control Parameters}

\begin{enumerate}
 \item X2A Current PSET   nn.n A

\color{red}
Already checked on Analog Control Parameters section of System Control Testing.
\color{black}

 \item X2A Current PLOW   nn.n A

\color{red}
Already checked on Analog Control Parameters section of System Control Testing.
\color{black}

 \item X2A Current PHIGH  nn.n A

\color{red}
Already checked on Analog Control Parameters section of System Control Testing.
\color{black}

 \item X2A Current PDiff n.n A

\color{red}
Already checked on Analog Control Parameters section of System Control Testing.
\color{black}

 \item X2A Current PSEN  n

\color{red}
Already checked on Analog Control Parameters section of System Control Testing.
\color{black}

\end{enumerate}

\paragraph{Analog Display Parameters}

\begin{enumerate}
 \item X2A Current PSET   nn.n A

\color{red}
Already checked on Analog Control Parameters section of System Control Testing.
\color{black}

 \item X2A Current PREAD  nn.n A

\color{red}
Already checked on Analog Control Parameters section of System Control Testing.
\color{black}

 \item X2A Current PLOW   nn.n A

\color{red}
Already checked on Analog Control Parameters section of System Control Testing.
\color{black}

 \item X2A Current PHIGH  nn.n A

\color{red}
Already checked on Analog Control Parameters section of System Control Testing.
\color{black}

 \item X2A Current PDiff n.n A

\color{red}
Already checked on Analog Control Parameters section of System Control Testing.
\color{black}

 \item X2A Current PSEN  n

\color{red}
Already checked on Analog Control Parameters section of System Control Testing.
\color{black}

\end{enumerate}

\subsubsection{Y2A Steering Magnet (Y2A) Device Operations} \label{sect:cyc-op-interface-ops-terminal-device-ops-y2a}

\paragraph{Title} \label{sect:cyc-op-interface-ops-terminal-device-ops-y2a-title}

The title of the system will be Y2A.  The color of the title bar will be teal.

\paragraph{State Controls}

\begin{enumerate}
\item None
\end{enumerate}

\paragraph{State Indicators}

\begin{enumerate}
 \item None
\end{enumerate}

\paragraph{Analog Control Parameters}

\begin{enumerate}
 \item Y2A Current PSET   nn.n A

\color{red}
See X2A
\color{black}

 \item Y2A Current PLOW   nn.n A

\color{red}
See X2A
\color{black}

 \item Y2A Current PHIGH  nn.n A

\color{red}
See X2A
\color{black}

 \item Y2A Current PDiff n.n A

\color{red}
See X2A
\color{black}

 \item Y2A Current PSEN  n

\color{red}
See X2A
\color{black}

\end{enumerate}

\paragraph{Analog Display Parameters}

\begin{enumerate}
 \item Y2A Current PSET   nn.n A

\color{red}
See X2A
\color{black}

 \item Y2A Current PREAD  nn.n A

\color{red}
See X2A.
\color{black}

 \item Y2A Current PLOW   nn.n A

\color{red}
See X2A.
\color{black}

 \item Y2A Current PHIGH  nn.n A

\color{red}
See X2A.
\color{black}

 \item Y2A Current PDiff n.n A

\color{red}
See X2A.
\color{black}

 \item Y2A Current PSEN  n

\color{red}
See X2A.
\color{black}

\end{enumerate}

\subsubsection{X3A Steering Magnet (X3A) Device Operations} \label{sect:cyc-op-interface-ops-terminal-device-ops-x3a}

\paragraph{Title} \label{sect:cyc-op-interface-ops-terminal-device-ops-x3a-title}

The title of the system will be X3A.  The color of the title bar will be teal.

\paragraph{State Controls}

\begin{enumerate}
\item None
\end{enumerate}

\paragraph{State Indicators}

\begin{enumerate}
 \item None
\end{enumerate}

\paragraph{Analog Control Parameters}

\begin{enumerate}
 \item X3A Current PSET   nn.n A

\color{red}
See X2A.
\color{black}

 \item X3A Current PLOW   nn.n A

\color{red}
See X2A.
\color{black}

 \item X3A Current PHIGH  nn.n A

\color{red}
See X2A.
\color{black}

 \item X3A Current PDiff n.n A

\color{red}
See X2A.
\color{black}

 \item X3A Current PSEN  n

\color{red}
See X2A.
\color{black}

\end{enumerate}

\paragraph{Analog Display Parameters}

\begin{enumerate}
 \item X3A Current PSET   nn.n A

\color{red}
See X2A.
\color{black}

 \item X3A Current PREAD  nn.n A

\color{red}
See X2A.
\color{black}

 \item X3A Current PLOW   nn.n A

\color{red}
See X2A.
\color{black}

 \item X3A Current PHIGH  nn.n A

\color{red}
See X2A.
\color{black}

 \item X3A Current PDiff n.n A

\color{red}
See X2A.
\color{black}

 \item X3A Current PSEN  n

\color{red}
See X2A.
\color{black}

\end{enumerate}

\subsubsection{Y3A Steering Magnet (Y3A) Device Operations} \label{sect:cyc-op-interface-ops-terminal-device-ops-y3a}

\paragraph{Title} \label{sect:cyc-op-interface-ops-terminal-device-ops-y3a-title}

The title of the system will be Y3A.  The color of the title bar will be teal.

\paragraph{State Controls}

\begin{enumerate}
\item None
\end{enumerate}

\paragraph{State Indicators}

\begin{enumerate}
 \item None
\end{enumerate}

\paragraph{Analog Control Parameters}

\begin{enumerate}
 \item Y3A Current PSET   nn.n A

\color{red}
See X2A.
\color{black}

 \item Y3A Current PLOW   nn.n A

\color{red}
See X2A.
\color{black}

 \item Y3A Current PHIGH  nn.n A

\color{red}
See X2A.
\color{black}

 \item Y3A Current PDiff n.n A

\color{red}
See X2A.
\color{black}

 \item Y3A Current PSEN  n

\color{red}
See X2A.
\color{black}

\end{enumerate}

\paragraph{Analog Display Parameters}

\begin{enumerate}
 \item Y3A Current PSET   nn.n A

\color{red}
See X2A.
\color{black}

 \item Y3A Current PREAD  nn.n A

\color{red}
See X2A.
\color{black}

 \item Y3A Current PLOW   nn.n A

\color{red}
See X2A.
\color{black}

 \item Y3A Current PHIGH  nn.n A

\color{red}
See X2A.
\color{black}

 \item Y3A Current PDiff n.n A

\color{red}
See X2A.
\color{black}

 \item Y3A Current PSEN  n

\color{red}
See X2A.
\color{black}

\end{enumerate}





\chapter{SM23A Status Display Testing}

\subsection{Beamline FC1 to Target system} \label{sect:cyc-op-interface-status-terminal-display-contents-beamline-target}

The Beamline to FC1 system includes the xxxx.

The Beamline FC1 to Target status screen pops up in the status terminal when the button "Beam Line FC1 to Target" is pushed on the CCC. The status screen is tabbed display such that the status of FC1-to-FC3, FC3-to-Target, Beamline B, and Beamline C may be displayed.  The switching magnet is shown for all 4 states, while the bending magnet is shown for only Beamline B and Beamline C.  When Beamline A is selected by the operator on the CCC, the tab for FC1-to-FC3 is selected. When Beamline B or C is selected, the corresponding tab in status dispay is selected. The user always has the ability to select the desired tab with the mouse pointer.

\subsubsection{Title}\label{sect:cyc-op-interface-status-terminal-display-contents-beamline-target-title}

The title of the display is "Beamline Control: FC1 to Target".  The color of the title bar is forest green.

\subsection{State Monitors} \label{sect:cyc-op-interface-status-beamline-fc1totarget-state-monitors}

\begin{enumerate}
 \item X2A/X3A/Y2A/Y3A (ON,OFF)

\color{red}
Already checked on State Controls section of System Control Testing.
\color{black}

 \item X2A/X3A/Y2A/Y3A Local

\color{red}
Already checked on State Monitor section of System Control Testing.
\color{black}

 \item X2A/X3A/Y2A/Y3A Initializing

\color{red}
Already checked on State Controls section of System Control Testing.
\color{black}

 \item X2A/X3A/Y2A/Y3A Shutting Down

\color{red}
Already checked on State Controls section of System Control Testing.
\color{black}

 \item SM23A +24 V Control Voltage Low

\color{red}
Not in control system.
\color{black}

 \item X2A Circuit Resistance out of Tolerance

\color{red}
Not in control system.
\color{black}

 \item X3A Circuit Resistance out of Tolerance

\color{red}
Not in control system.
\color{black}

 \item Y2A Circuit Resistance out of Tolerance

\color{red}
Not in control system.
\color{black}

 \item Y3A Circuit Resistance out of Tolerance

\color{red}
Not in control system.
\color{black}

 \item GCC (ON,OFF)
 \item GCC Local
 \item GCC Initializing
 \item GCC Shutting Down
 \item GCC +24 V Control Voltage Low
 \item GCC Circuit Resistance out of Tolerance
\end{enumerate}

\subsubsection{Device Interlocks}\label{sect:cyc-op-interface-status-beamline-fc1totarget-state-monitors-device-interlocks}

Device Interlocks:
(Occurence of these will turn off the corresponding device and prevent it from being turned back on until addressed)

\begin{enumerate}
 \item X2A/X3A/Y2A/Y3A Ground Fault - Reset Locally to Remove latch - No Remote Reset

\color{red}
Already checked in Device section of System Control Testing.
\color{black}

 \item X2A/Y2A Over Temperature - Reset to Remove latch

\color{red}
Already checked in Device section of System Control Testing.
\color{black}

 \item X3A/Y3A Over Temperature - Reset to Remove latch

\color{red}
Already checked in Device section of System Control Testing.
\color{black}

 \item SWM Magnet Cooling - Reset to Remove latch
 \item GCC Ground Fault - Reset Locally to Remove latch - No Remote Reset
 \item GCC Flow and Temperature - Reset to Remove latch
\end{enumerate}

\subsubsection{Particle Beam Interlocks}\label{sect:cyc-op-interface-status-beamline-fc1totarget-state-monitors-beam-interlocks}

Particle Beam Interlocks:
(Occurrence of interlock will prevent RF system from attempting to accelerate a particle beam)

\begin{enumerate}
 \item X2A/X3A/Y2A/Y3A Initializing - Non-latching

\color{red}
Already checked in Particle Beam Interlock section of System Control Testing.
\color{black}

 \item X2A/X3A/Y2A/Y3A Off - Non-latching

\color{red}
Already checked in Particle Beam Interlock section of System Control Testing.
\color{black}

 \item X2A/X3A/Y2A/Y3A Shutting Down - Non-latching

\color{red}
Already checked in Particle Beam Interlock section of System Control Testing.
\color{black}

 \item PSX2A Current $PREAD \geq PHIGH$ - Non-latching

\color{red}
Already checked in Particle Beam Interlock section of System Control Testing.
\color{black}

 \item PSX2A Current $PREAD \leq PLOW$ - Non-latching

\color{red}
Already checked in Particle Beam Interlock section of System Control Testing.
\color{black}

 \item PSX2A Current $\mid$PREAD-PSET$\mid$  $\geq$ PDiff - Non-latching

\color{red}
Already checked in Particle Beam Interlock section of System Control Testing.
\color{black}

 \item PSX2A Communication Fault - Initialize to remove latch

\color{red}
Already checked in Particle Beam Interlock section of System Control Testing.
\color{black}

 \item PSY2A Current $PREAD \geq PHIGH$ - Non-latching

\color{red}
See X2A.
\color{black}

 \item PSY2A Current $PREAD \leq PLOW$ - Non-latching

\color{red}
See X2A.
\color{black}

 \item PSY2A Current $\mid$PREAD-PSET$\mid$  $\geq$ PDiff - Non-latching

\color{red}
See X2A.
\color{black}

 \item PSY2A Communication Fault - Initialize to remove latch

\color{red}
See X2A.
\color{black}

 \item PSX3A Current $PREAD \geq PHIGH$ - Non-latching

\color{red}
See X2A.
\color{black}

 \item PSX3A Current $PREAD \leq PLOW$ - Non-latching

\color{red}
See X2A.
\color{black}

 \item PSX3A Current $\mid$PREAD-PSET$\mid$  $\geq$ PDiff - Non-latching

\color{red}
See X2A.
\color{black}

 \item PSX3A Communication Fault - Initialize to remove latch

\color{red}
See X2A.
\color{black}

 \item PSY3A Current $PREAD \geq PHIGH$ - Non-latching

\color{red}
See X2A.
\color{black}

 \item PSY3A Current $PREAD \leq PLOW$ - Non-latching

\color{red}
See X2A.
\color{black}

 \item PSY3A Current $\mid$PREAD-PSET$\mid$  $\geq$ PDiff - Non-latching

\color{red}
See X2A.
\color{black}

 \item PSY3A Communication Fault - Initialize to remove latch

\color{red}
See X2A.
\color{black}

 \item X2A/X3A/Y2A/Y3A Watchdog - Initialize to remove latch

\color{red}
Already checked in Particle Beam Interlock section of System Control Testing.
\color{black}

 \item GCC Initializing - Non-latching
 \item GCC Off - Non-latching
 \item GCC Shutting Down - Non-latching
 \item GCC Current $PREAD \geq PHIGH$ - Non-latching
 \item GCC Current $PREAD \leq PLOW$ - Non-latching
 \item GCC Current $\mid$PREAD-PSET$\mid$  $\geq$ PDiff - Non-latching
 \item GCC Communication Fault - Initialize to remove latch
 \item GCC Watchdog - Initialize to remove latch
\end{enumerate}


\subsection{Safety}\label{sect:cyc-op-interface-status-beamline-fc1totarget-safety}

Loss of control of the beamline magnets will result in loss of control of the particle beam if one is being produced.  The primary protection against this is that the control system monitors the output current of the control magnets. If the output current is out of tolerance or in question (e.g., see section \ref{sect:cyc-equip-ctl-beamline-sm23a-state-monitors-beam-interlocks} for a description) the control system will shut off the particle beam as described in section \ref{sect:cyc-equip-ctl-safety-sys-control-beam-control}.  The back up for this is provided by the particle beam hardwired safety interlock system (HSIS) (section \ref{sect:cyc-equip-ctl-safety-sys-hsis-beam}) that monitors the particle beam position and beam losses described in section \ref{ch:cyc-equip-ctl-beam-diagnostics}.


\subsection{Analog Control Parameters}\label{sect:cyc-op-interface-status-beamline-fc1totarget-analog-control}

\begin{enumerate}
 \item PSX2A Current PSET  n.nn A

\color{red}
Already checked on Analog Control Parameters section of System Control Testing.
\color{black}

 \item PSX2A Current PLOW  n.nn A

\color{red}
Already checked on Analog Control Parameters section of System Control Testing.
\color{black}

 \item PSX2A Current PHIGH n.nn A

\color{red}
Already checked on Analog Control Parameters section of System Control Testing.
\color{black}

 \item PSX2A Current PDiff n.n A

\color{red}
Already checked on Analog Control Parameters section of System Control Testing.
\color{black}

 \item PSX2A Current PSEN  n

\color{red}
Already checked on Analog Control Parameters section of System Control Testing.
\color{black}

 \item PSY2A Current PSET  n.nn A

\color{red}
See X2A
\color{black}

 \item PSY2A Current PLOW  n.nn A

\color{red}
See X2A
\color{black}

 \item PSY2A Current PHIGH n.nn A

\color{red}
See X2A
\color{black}

 \item PSY2A Current PDiff n.n A

\color{red}
See X2A
\color{black}

 \item PSY2A Current PSEN  n

\color{red}
See X2A
\color{black}

 \item PSX3A Current PSET  n.nn A

\color{red}
See X2A
\color{black}

 \item PSX3A Current PLOW  n.nn A

\color{red}
See X2A
\color{black}

 \item PSX3A Current PHIGH n.nn A

\color{red}
See X2A
\color{black}

 \item PSX3A Current PDiff n.n A

\color{red}
See X2A
\color{black}

 \item PSX3A Current PSEN  n

\color{red}
See X2A
\color{black}

 \item PSY3A Current PSET  n.nn A

\color{red}
See X2A
\color{black}

 \item PSY3A Current PLOW  n.nn A

\color{red}
See X2A
\color{black}

 \item PSY3A Current PHIGH n.nn A

\color{red}
See X2A
\color{black}

 \item PSY3A Current PDiff n.n A

\color{red}
See X2A
\color{black}

 \item PSY3A Current PSEN  n

\color{red}
See X2A
\color{black}

 \item GCC Current PSET  n.nn A
 \item GCC Current PLOW  n.nn A
 \item GCC Current PHIGH n.nn A
 \item GCC Current PDiff n.n A
 \item GCC Current PSEN  n
\end{enumerate}


\subsection{Parameter Limits} \label{sect:cyc-op-interface-status-beamline-fc1totarget-analog-control-limits}

\begin{enumerate}
 \item PSX2A Current MinLimit = -1.3 A

\color{red}
Already checked in Parameter Limits section of System Control Testing.
\color{black}

 \item PSX2A Current MaxLimit = 1.3 A

\color{red}
Already checked in Parameter Limits section of System Control Testing.
\color{black}

 \item PSX2A Current PLOWWarn = PLOW + 0.5 A

\color{red}
Already checked in Parameter Limits section of System Control Testing.
\color{black}

 \item PSX2A Current PHIGHWarn = PHIGH - 0.5 A

\color{red}
Already checked in Parameter Limits section of System Control Testing.
\color{black}

 \item PSY2A Current MinLimit = -1.3 A

\color{red}
See X2A
\color{black}

 \item PSY2A Current MaxLimit = 1.3 A

\color{red}
See X2A
\color{black}

 \item PSY2A Current PLOWWarn = PLOW + 0.5 A

\color{red}
See X2A
\color{black}

 \item PSY2A Current PHIGHWarn = PHIGH - 0.5 A

\color{red}
See X2A
\color{black}

 \item PSX3A Current MinLimit = -1.3 A

\color{red}
See X2A
\color{black}

 \item PSX3A Current MaxLimit = 1.3 A

\color{red}
See X2A
\color{black}

 \item PSX3A Current PLOWWarn = PLOW + 0.5 A

\color{red}
See X2A
\color{black}

 \item PSX3A Current PHIGHWarn = PHIGH - 0.5 A

\color{red}
See X2A
\color{black}

 \item PSY3A Current MinLimit = -1.3 A

\color{red}
See X2A
\color{black}

 \item PSY3A Current MaxLimit = 1.3 A

\color{red}
See X2A
\color{black}

 \item PSY3A Current PLOWWarn = PLOW + 0.5 A

\color{red}
See X2A
\color{black}

 \item PSY3A Current PHIGHWarn = PHIGH - 0.5 A

\color{red}
See X2A
\color{black}

 \item GCC Current MinLimit = -1.3 A
 \item GCC Current MaxLimit = 1.3 A
 \item GCC Current PLOWWarn = PLOW + 0.5 A
 \item GCC Current PHIGHWarn = PHIGH - 0.5 A
\end{enumerate}

\subsection{Analog Display Parameters} \label{sect:cyc-op-interface-status-beamline-fc1totarget-analog-display}

\begin{enumerate}
 \item PSX2A Current PSET  n.nn A

\color{red}
Already checked on Analog Control Parameters section of System Control Testing.
\color{black}

 \item PSX2A Current PLOW  n.nn A

\color{red}
Already checked on Analog Control Parameters section of System Control Testing.
\color{black}

 \item PSX2A Current PHIGH n.nn A

\color{red}
Already checked on Analog Control Parameters section of System Control Testing.
\color{black}

 \item PSX2A Current PDiff n.nn A

\color{red}
Already checked on Analog Control Parameters section of System Control Testing.
\color{black}

 \item PSX2A Current PSEN  n

\color{red}
Already checked on Analog Control Parameters section of System Control Testing.
\color{black}

 \item PSX2A Voltage PREAD n.nn V

\color{red}
Not in control system.
\color{black}

 \item X2A Resistance PREAD n.n $\Omega$

\color{red}
Not in control system.
\color{black}

 \item X2A Resistance PLOW n.n $\Omega$

\color{red}
Not in control system.
\color{black}

 \item X2A Resistance PHIGH n.n $\Omega$

\color{red}
Not in control system.
\color{black}

 \item PSX3A Current PSET  n.nn A

\color{red}
See X2A.
\color{black}

 \item PSX3A Current PLOW  n.nn A

\color{red}
See X2A.
\color{black}

 \item PSX3A Current PHIGH n.nn A

\color{red}
See X2A.
\color{black}

 \item PSX3A Current PDiff n.nn A

\color{red}
See X2A.
\color{black}

 \item PSX3A Current PSEN  n

\color{red}
See X2A.
\color{black}

 \item PSX3A Voltage PREAD n.nn V

\color{red}
Not in control system.
\color{black}

 \item X3A Resistance PREAD n.n $\Omega$

\color{red}
Not in control system.
\color{black}

 \item X3A Resistance PLOW n.n $\Omega$

\color{red}
Not in control system.
\color{black}

 \item X3A Resistance PHIGH n.n $\Omega$

\color{red}
Not in control system.
\color{black}

 \item PSY2A Current PSET  n.nn A

\color{red}
See X2A.
\color{black}

 \item PSY2A Current PLOW  n.nn A

\color{red}
See X2A.
\color{black}

 \item PSY2A Current PHIGH n.nn A

\color{red}
See X2A.
\color{black}

 \item PSY2A Current PDiff n.nn A

\color{red}
See X2A.
\color{black}

 \item PSY2A Current PSEN  n

\color{red}
See X2A.
\color{black}

 \item PSY2A Voltage PREAD n.nn V

\color{red}
Not in control system.
\color{black}

 \item Y2A Resistance PREAD n.n $\Omega$

\color{red}
Not in control system.
\color{black}

 \item Y2A Resistance PLOW n.n $\Omega$

\color{red}
Not in control system.
\color{black}

 \item Y2A Resistance PHIGH n.n $\Omega$

\color{red}
Not in control system.
\color{black}

 \item PSY3A Current PSET  n.nn A

\color{red}
See X2A.
\color{black}

 \item PSY3A Current PLOW  n.nn A

\color{red}
See X2A.
\color{black}

 \item PSY3A Current PHIGH n.nn A

\color{red}
See X2A.
\color{black}

 \item PSY3A Current PDiff n.nn A

\color{red}
See X2A.
\color{black}

 \item PSY3A Current PSEN  n

\color{red}
See X2A..
\color{black}

 \item PSY3A Voltage PREAD n.nn V

\color{red}
Not in control system.
\color{black}

 \item Y3A Resistance PREAD n.n $\Omega$

\color{red}
Not in control system.
\color{black}

 \item Y3A Resistance PLOW n.n $\Omega$

\color{red}
Not in control system.
\color{black}

 \item Y3A Resistance PHIGH n.n $\Omega$

\color{red}
Not in control system.
\color{black}

 \item GCC Current PSET  n.nn A
 \item GCC Current PLOW  n.nn A
 \item GCC Current PHIGH n.nn A
 \item GCC Current PDiff n.nn A
 \item GCC Current PSEN  n
 \item GCC Voltage PREAD n.nn V
 \item GCC Resistance PREAD n.n $\Omega$
 \item GCC Resistance PLOW n.n $\Omega$
 \item GCC Resistance PHIGH n.n $\Omega$
\end{enumerate}

\subsection{Implementation Notes} \label{sect:cyc-equip-ctl-beamline-sm23a-implementation}

The SM23A Steering Magnets and Gantry Correction Coil power supplies are commercial Kikusui DC supplies.  Digital control and monitoring of this power supply is via the MOD 1 PLC.  Analog control and monitoring are done by the control system using by way of Ethernet/GPIB via a Agilent E5810A GPIB gateway.


\end{document}
