\documentclass[11pt]{book}		% drafthead style seems not to work w\book
\usepackage{graphicx}
\usepackage{amsmath}
\usepackage{color}
%\usepackage{fancyhdr}
%\pagestyle{fancy}
%\lhead{\today}
\setlength{\oddsidemargin}{0.50in}	% Note binding-side margin is wider,
\setlength{\evensidemargin}{0.00in}	% unlike Lamport's defaults
\setlength{\topmargin}{0.0in}
\setlength{\textheight}{8.0in}
\setlength{\textwidth}{6.0in}
\setlength{\parindent}{0.0in}
\setlength{\parskip}{0.5cm}



\title{CLINICAL NEUTRON THERAPY SYSTEM\\
	Control System Specification 1.0\\[1.0cm]}
%         Beam Diagnostics and Status Display \\[1.0cm]}
%        Includes Operator Interface Overview, Status Terminal, Operatons Terminal, Equipment Control Overview, Magnet System Control, Extraction System Control, Beam Diagnostics System Control, Safety System, Control System\\[1.0cm]}


\author{Robert Emery\\
	Ruedi Risler\\
	Dave Reid \\
	Mat Hicks \\
        Jonathan Jacky}
%	\\ 
%	Jonathan Unger \\
%	Stan Brossard \\ [0.5cm]
%	Radiation Oncology Department \\
%	University of Washington\\
%	Seattle, WA  98195 \\[0.5cm]
%	Technical Report 92-05-01

\date{\today}

\begin{document}









\chapter{GCC System Control Testing}

The Gantry Correction Coil Power Supply delivers up to 2.5 A and 40 V to modify the magnetic fields in the Gantry Dipole. These fields transport the beam around the 90 degree bend to the target in the Gantry head. The correction coil was added after difficulties were found transporting the beam around this bend. Difficulities were later determined to be the result of incoming beam misalignment and use of the GCC subsequently stopped.  These power supplies operate in a current controlled mode with a minimum 12 bit I/O resolution, an output reproducibility of $\pm$.117 A and a longterm output stability of $\pm$ .02 mA over 24 hours.

\subsection{State Controls} \label{sect:cyc-equip-ctl-beamline-gcc-state-controls}
(See section \ref{sect:cyc-equip-ctl-definitions} for control definitions not described below)

\begin{enumerate}

 \item (ON,OFF)

\color{red}
1) Press the On button in the operations console. Observe the following: 
	Does the PS come on? 
	Does GCC:On:Status go from zero to 1 when you do this? 
	Does GCC:Off:Status go from 1 to 0 when you do this? 
	Does GCC:Condition:Initializing:Status get set to 1 then 0?
	Does GCC:Reset:Interlocks:Write go to 1, then 0 again?

2) Set GCC:CommError:Status to false, then turn on and see if this clears.

3) Set PSET to 1.0 A for GCC. Turn the PS off, and see if the "GCC:Curr:Set" goes to zero before "GCC:On:Status". While this occurs, GCC:Condition:ShuttingDown:Status should go to 1 temporarily. Then, turn on the PS back on, and ensure that "GCC
:Curr:Set" returns to 1.0 A.

4) Does pressing the Off button at the operations console turn off the PS? When you press off, does the "Shutting Down" light flash in both the operations and status display?

5) On the Operations Console, and Status display, observe that PSET says zero for GCC when the PS is off.

6) Place the device in local mode, and attempt to turn on the PS. You shouldn't be able to do this.

7) When you turn on the device locally, make sure GCC:Condition:Initializing:Status gets set to 1 then 0.


\color{black}

 \item Initialize

\color{red}

Press the Initialize button in the operations console and observe the following: 
	Does the PV GCC:Condition:Initializing:Status go from 0 to 1, then back to 0 during this process? 
	Ensure that it flashes "Initializing" in Yellow on the operations and status display screen. 
	Make sure you see the PV GCC:Init:Status goes from 0 to 1 then back to 0.

Try setting Set GCC:CommError:Status to false, then press Initialize and see if it goes back to true.
Turn the PV RF:HighPowerOn:Status from 0 to 1, and verify that you cannot press the Initialize button. 

I tried this: Then, run a caput GCC:Init:String.PROC 1, and verify that GCC:Init:Status stay at 0, and an Initization aborted b/c the beam is on message comes up on the CCC.

But then realized that the Input field of the calc record does not want to update when you do this. I am not sure why. I need to investigate this further. So i modify RF:HighPowerOn:Status with the adjust button on a probe, then the only way it will update on the calc record in GCCInitialize.vdb, is if I restart the IOC. Otherwise, it won't read it. Maybe the record needs to have been processed? Does this only work for a CA link? Need to investigate this further.
\color{black}


 \item Reset

\color{red}

Press the Reset button on the operations console, and observe the PV GCC:Reset:Interlocks:Write go to 1, then 0 again. Other then that, I don't know how to force the temperature interlocks to trigger.

\color{black}

\end{enumerate}

\subsubsection{Standby 1 to Standby 2 Transition}

When the system is commanded to transition to the Standby 2 state as described in section \ref{sect:cyc-equip-ctl-controls-system-coordination-standby}, PSGCC is turned on.

\color{red}

To Simulate going to SB2, set PSET not equal to 0, then place a 1 on the field, BeamlineControl:Standby2:BeamlineA:GCC.PROC. This should turn on gcc. Also, verify that GCC:SubsystemOKSB2:Status goes from 0 to 1 after this happens.Then, turn off the GCC, and set pset = 0. Then write a 1 to BeamlineControl:Standby2:BeamlineA:GCC.PROC and verify that this does not turn on the GCC. Also, verify that GCC:SubsystemOKSB2:Status goes from 0 to 1 after this happens. Then, write pset not equal to zero, and verify that GCC:SubsystemOKSB2:Status goes from 0 to 1 after this happens.

Also, when line A and standby 2 are selected, gcc should turn on. When any other line is selected in sb2, gcc should turn off.

\color{black}

\subsubsection{Standby 2 to Standby 1 Transition}

When the system is commanded to transition to the Standby 2 state as described in section \ref{sect:cyc-equip-ctl-controls-system-coordination-standby}, PSGCC is turned off.

\color{red}

To Simulate going to SB1, you place a 1 on the field, BeamlineControl:GotoStandby1.PROC. This should turn off gcc.  Verify that this happens.

\color{black}


\subsection{State Monitors} \label{sect:cyc-equip-ctl-beamline-gcc-state-monitors}
(See section \ref{sect:cyc-equip-ctl-definitions} for state definitions not described below)

\begin{enumerate}
 \item GCC (ON,OFF)

\color{red}
Already checked this in State Controls above.
\color{black}

 \item GCC Local

\color{red}
Press the Local button the PS, and see if the "Local" buttons lights up in both the operations terminal and status terminal.
\color{black}

 \item GCC Initializing

\color{red}
Already checked this in State Controls above.
\color{black}

 \item GCC Shutting Down

\color{red}
Already checked this in State Controls above.
\color{black}

 \item +24 V Control Voltage Low

\color{red}
Not in control system.
\color{black}

 \item GCC Circuit Resistance out of Tolerance

\color{red}
Not in control system.
\color{black}

\end{enumerate}

\subsubsection{Device Interlocks}

GCC Device Interlocks:
(Occurrence of interlock will turn off the GCC PS and not allow it to be turned on unless otherwise noted)

\begin{enumerate}
 \item GCC Ground Fault - Reset Locally to Remove latch - No Remote Reset

\color{red}
In the modicon PLC, force read coil 10250 Off, and see if this is displayed in status display. Also "Device Interlock" should appear in red at above the operations console. I have not figured out a way to check to see if an actual ground fault would trigger this coil, or not. The reset command cannot fully be tested in this sense, eithier.
\color{black}

 \item GCC Flow and Temperature - Reset to Remove latch

\color{red}
In the modicon PLC, force read coil 10254 On, and see if this is displayed in status display. Also "Device Interlock" should appear in red at above the operations console. I have not figured out a way to check to see if an actual ground fault would trigger this coil, or not. The reset command cannot fully be tested in this sense, eithier.
\color{black}


\end{enumerate}

\subsubsection{Particle Beam Interlocks}

GCC Particle Beam Interlocks:
(Occurrence of interlock will prevent RF system from attempting to accelerate a particle beam)

\begin{enumerate}
 \item GCC Initializing - Non-latching

\color{red}
Write GCC:Condition:Initializing:Status from 0 to 1, and verify that GCC:SubsystemOKSB2:Status goes from 1 to 0.
\color{black}

 \item GCC Off - Non-latching

\color{red}
If the PS is off, and PSET is 0, then GCC:SubsystemOKSB2:Status should be 1.
\color{black}

 \item GCC Shutting Down - Non-latching

\color{red}
Write GCC:Condition:ShuttingDown:Status to 1, and verify that GCC:SubsystemOKSB2:Status goes from 1 to 0.
\color{black}

 \item GCC Current $PREAD \geq PHIGH$ - Non-latching

\color{red}
With PSET=0.5, set PHIGH to 0, and verify that GCC:SubsystemOKSB2:Status goes from 1 to 0. Verify the appropriate changes in the PREAD display in operations console and status display.
\color{black}

 \item GCC Current $PREAD \leq PLOW$ - Non-latching

\color{red}
With PSET=0.5, set PLOW to .6, and verify that GCC:SubsystemOKSB2:Status goes from 1 to 0.Verify the appropriate changes in the PSET display in operations console and status display.
\color{black}

 \item GCC Current $\mid$PREAD-PSET$\mid$  $\geq$ PDiff - Non-latching

\color{red}
With PSET=0.5, set PDIFF to 0, and verify that SM23A:SubsystemOKSB2:Status goes from 1 to 0 (may have to wait). Verify the appropriate changes in the PSET display in operations console and status display. This is a hard one to get off b/c the PS's are so stable and precise. I had trouble getting the PDIFF to actually be read.
\color{black}

 \item GCC Communication Fault - Initialize to remove latch

\color{red}
Write GCC:CommError:Status from 0 to 1. Does GCC:SubsystemOKSB2:Status goes from 1 to 0? Does a CommError Fushia colored light come on in the operations console and status display?
\color{black}

 \item GCC Watchdog - Initialize to remove latch

\color{red}
The HB should always be going, even if the PS is off. Turn off the PS, and verify that the HB is going (GCC:HeartbeatOK:Status) with and without PSET=0.
\color{black}


\end{enumerate}


\subsection{Safety}

If used, loss of control of the PSGCC would result in loss of control of the particle beam if one is being produced.  The primary protection against this is that the control system monitors the output current of the PSGCC.  If the output current is out of tolerance or in question, as described in section \ref{sect:cyc-equip-ctl-beamline-gcc-state-monitors-beam-interlocks}, the control system will shut off the particle beam as described in section \ref{sect:cyc-equip-ctl-safety-sys-control-beam-control}.  The back up for this is provided by the particle beam hardwired safety interlock system (HSIS) (section \ref{sect:cyc-equip-ctl-safety-sys-hsis-beam}) that monitors the particle beam position and beam losses described in section \ref{ch:cyc-equip-ctl-beam-diagnostics}.


\color{red}

Bring up everything into SB2, and select beamline A in test. Make sure everything is okay, so that you can run beam down the A line onto target. Check that all the appropriate buttons are lit up, particularly the ``FC1 to Target'' button light should be on, and the FC1 ``Interlock'' light should be off. With the RF On, and FC1 Open (as well as FC2 and FC3), but NO CATHODE CURRENT, change PLOW below PSET and observe that the RF shuts off, and FC1 gets put in. Then observe the ``Interlock'' light is on next to FC1, and that the light on the ``FC1 to Tartget'' button turns off. Try opening FC1 and beam plug, and verify that they won't open. Turn the RF back on, with FC1 in. With the RF on, verify that you still cannot open FC1 or beamplug. Then change PLOW for SM23A back to the appropriate value and verify that the FC1 ``Interlock'' light goes away, the ``FC1 to Target'' button light turns on, the beam plug opens, and that now you can open the FC1 with the RF On.

Do this identical procedure for beamline A, in treat mode.

\color{black}


\subsection{Analog Control Parameters}

\begin{enumerate}
 \item GCC Current PSET  n.nn A

\color{red}
Set a value for PSET, and verify that it appears on the PS, in operations console and status display, and on PREAD (in operations console and status display).
\color{black}

 \item GCC Current PLOW  n.nn A

\color{red}
Already tested in Particle Beam Interlock section.
\color{black}

 \item GCC Current PHIGH n.nn A

\color{red}
Already tested in Particle Beam Interlock section.
\color{black}

 \item GCC Current PDiff n.n A

\color{red}
Already tested in Particle Beam Interlock section.
\color{black}

 \item GCC Current PSEN  n

\color{red}
Change PSEN from 1 to 10, and verify that the tuning knob sensivity has changed in the tuning module. Here, you should also check that you can drag GCC into the tuning module, and control the PSET parameter with the knob. Move the knob, and observe the pset value on status display.
\color{black}

\end{enumerate}


\subsection{Parameter Limits}

\begin{enumerate}
 \item GCC Current MinLimit = -1.3 A

\color{red}
This is hard encoded into the PS. If you do a software encode, it resets after it is turned off. The hardware encoded limit forces you to turn off the PS and turn it back on again if you go over. The operator is not notified of this. Should we keep it in?
\color{black}

 \item GCC Current MaxLimit = 1.3 A

\color{red}
This is hard encoded into the PS. If you do a software encode, it resets after it is turned off. The hardware encoded limit forces you to turn off the PS and turn it back on again if you go over. The operator is not notified of this. Should we keep it in?
\color{black}

 \item GCC Current PLOWWarn = PLOW + 0.5 A

\color{red}
With PSET=-0.51, and PLOW set to -1.0, verify the appropriate changes in the PSET display in operations console (both tuning modules) and status display.
\color{black}

 \item GCC Current PHIGHWarn = PHIGH - 0.5 A

\color{red}
With PSET=0.51, and PHIGH set to 1.0, verify the appropriate changes in the PSET display in operations console (both tuning modules) and status display.
\color{black}

\end{enumerate}

\subsection{Analog Display Parameters}

\begin{enumerate}
 \item GCC Current PSET  n.nn A

\color{red}
Tested in Analog Control Parameters.
\color{black}

 \item GCC Current PLOW  n.nn A

\color{red}
Tested in Analog Control Parameters.
\color{black}

 \item GCC Current PHIGH n.nn A

\color{red}
Tested in Analog Control Parameters.
\color{black}

 \item GCC Current PDiff n.nn A

\color{red}
Tested in Analog Control Parameters.
\color{black}

 \item GCC Current PSEN  n

\color{red}
Tested in Analog Control Parameters.
\color{black}

 \item GCC Voltage PREAD n.nn V

\color{red}
Not in control system.
\color{black}

 \item GCC Resistance PREAD n.n $\Omega$

\color{red}
Not in control system.
\color{black}

 \item GCC Resistance PLOW n.n $\Omega$

\color{red}
Not in control system.
\color{black}

 \item GCC Resistance PHIGH n.n $\Omega$

\color{red}
Not in control system.
\color{black}

\end{enumerate}

\subsection{Implementation Notes}

The GCC power supply is commercial Kikusui DC supplies.  Digital control and monitoring of this power supply is via the MOD 1 PLC.  Analog control and monitoring are done by the control system using by way of Ethernet/GPIB via a Agilent E5810A GPIB gateway.






\chapter{GCC Operations Console Testing}

\paragraph{Draggable Objects}

\begin{enumerate}
 \item ``GCC'' label. - Will assign GCC output current parameter to Tuning Module or open GCC Display in Device Operations region.

\color{red}
Already tested in Analog Control Parameters.
\color{black}

\end{enumerate}

\paragraph{State Controls}

\begin{enumerate}
 \item GCC (ON,OFF)

\color{red}
Already checked on State Controls of System Control Testing.
\color{black}

 \item GCC Initialize (Button is disabled if device is off or if RF in High Power mode)

\color{red}
Already checked on State Controls of System Control Testing.
\color{black}

\end{enumerate}

\paragraph{State Indicators}

\begin{enumerate}
 \item GCC (ON,OFF)

\color{red}
Already checked on State Controls of System Control Testing.
\color{black}

 \item GCC Local

\color{red}
Already checked on State Monotors of System Control Testing.
\color{black}

 \item GCC Initializing - Flashing Yellow

\color{red}
Already checked on State Controls of System Control Testing.
\color{black}

 \item GCC Device Interlock - Red

\color{red}
Already checked on Device Interlocks Section of System Control Testing.
\color{black}

 \item GCC Communication Fault - Fuchsia

\color{red}
Already checked on Particle Beam Interlocks of System Control Testing.
\color{black}

 \item GCC Process Heartbeat - Fuchsia

\color{red}
Already checked on Particle Beam Interlocks of System Control Testing.
\color{black}

\end{enumerate}

\subsubsection{GCC Steering Magnet (GCC) Device Operations}

\paragraph{Title} \label{sect:cyc-op-interface-ops-terminal-device-ops-gcc-title}

The title of the system will be GCC.  The color of the title bar will be teal.

\paragraph{State Controls}

\begin{enumerate}
\item None
\end{enumerate}

\paragraph{State Indicators}

\begin{enumerate}
 \item None
\end{enumerate}

\paragraph{Analog Control Parameters}

\begin{enumerate}
 \item GCC Current PSET   nn.n A

\color{red}
Already checked on Analog Control Parameters section of System Control Testing.
\color{black}

 \item GCC Current PLOW   nn.n A

\color{red}
Already checked on Analog Control Parameters section of System Control Testing.
\color{black}

 \item GCC Current PHIGH  nn.n A

\color{red}
Already checked on Analog Control Parameters section of System Control Testing.
\color{black}

 \item GCC Current PDiff n.n A

\color{red}
Already checked on Analog Control Parameters section of System Control Testing.
\color{black}

 \item GCC Current PSEN  n

\color{red}
Already checked on Analog Control Parameters section of System Control Testing.
\color{black}

\end{enumerate}

\paragraph{Analog Display Parameters}

\begin{enumerate}
 \item GCC Current PSET   nn.n A

\color{red}
Already checked on Analog Control Parameters section of System Control Testing.
\color{black}

 \item GCC Current PREAD  nn.n A

\color{red}
Already checked on Analog Control Parameters section of System Control Testing.
\color{black}

 \item GCC Current PLOW   nn.n A

\color{red}
Already checked on Analog Control Parameters section of System Control Testing.
\color{black}

 \item GCC Current PHIGH  nn.n A

\color{red}
Already checked on Analog Control Parameters section of System Control Testing.
\color{black}

 \item GCC Current PDiff n.n A

\color{red}
Already checked on Analog Control Parameters section of System Control Testing.
\color{black}

 \item GCC Current PSEN  n

\color{red}
Already checked on Analog Control Parameters section of System Control Testing.
\color{black}

\end{enumerate}






\chapter{GCC Status Display Testing}

The Beamline to FC1 system includes the xxxx.

The Beamline FC1 to Target status screen pops up in the status terminal when the button "Beam Line FC1 to Target" is pushed on the CCC. The status screen is tabbed display such that the status of FC1-to-FC3, FC3-to-Target, Beamline B, and Beamline C may be displayed.  The switching magnet is shown for all 4 states, while the bending magnet is shown for only Beamline B and Beamline C.  When Beamline A is selected by the operator on the CCC, the tab for FC1-to-FC3 is selected. When Beamline B or C is selected, the corresponding tab in status dispay is selected. The user always has the ability to select the desired tab with the mouse pointer.

\subsubsection{Title}\label{sect:cyc-op-interface-status-terminal-display-contents-beamline-target-title}

The title of the display is "Beamline Control: FC1 to Target".  The color of the title bar is forest green.

\subsection{State Monitors}

\begin{enumerate}
 \item X2A/X3A/Y2A/Y3A (ON,OFF)
 \item X2A/X3A/Y2A/Y3A Local
 \item X2A/X3A/Y2A/Y3A Initializing
 \item X2A/X3A/Y2A/Y3A Shutting Down
 \item SM23A +24 V Control Voltage Low
 \item X2A Circuit Resistance out of Tolerance
 \item X3A Circuit Resistance out of Tolerance
 \item Y2A Circuit Resistance out of Tolerance
 \item Y3A Circuit Resistance out of Tolerance
 \item GCC (ON,OFF)

\color{red}
Already checked on State Controls section of System Control Testing.
\color{black}

 \item GCC Local

\color{red}
Already checked on State Monitor section of System Control Testing.
\color{black}

 \item GCC Initializing

\color{red}
Already checked on State Controls section of System Control Testing.
\color{black}

 \item GCC Shutting Down

\color{red}
Already checked on State Controls section of System Control Testing.
\color{black}

 \item GCC +24 V Control Voltage Low

\color{red}
Not in control system.
\color{black}

 \item GCC Circuit Resistance out of Tolerance

\color{red}
Not in control system.
\color{black}

\end{enumerate}

\subsubsection{Device Interlocks}

Device Interlocks:
(Occurence of these will turn off the corresponding device and prevent it from being turned back on until addressed)

\begin{enumerate}
 \item X2A/X3A/Y2A/Y3A Ground Fault - Reset Locally to Remove latch - No Remote Reset
 \item X2A/Y2A Over Temperature - Reset to Remove latch
 \item X3A/Y3A Over Temperature - Reset to Remove latch
 \item SWM Magnet Cooling - Reset to Remove latch
 \item GCC Ground Fault - Reset Locally to Remove latch - No Remote Reset

\color{red}
Already checked in Device section of System Control Testing.
\color{black}

 \item GCC Flow and Temperature - Reset to Remove latch

\color{red}
Already checked in Device section of System Control Testing.
\color{black}

\end{enumerate}

\subsubsection{Particle Beam Interlocks}

Particle Beam Interlocks:
(Occurrence of interlock will prevent RF system from attempting to accelerate a particle beam)

\begin{enumerate}
 \item X2A/X3A/Y2A/Y3A Initializing - Non-latching
 \item X2A/X3A/Y2A/Y3A Off - Non-latching
 \item X2A/X3A/Y2A/Y3A Shutting Down - Non-latching
 \item PSX2A Current $PREAD \geq PHIGH$ - Non-latching
 \item PSX2A Current $PREAD \leq PLOW$ - Non-latching
 \item PSX2A Current $\mid$PREAD-PSET$\mid$  $\geq$ PDiff - Non-latching
 \item PSX2A Communication Fault - Initialize to remove latch
 \item PSY2A Current $PREAD \geq PHIGH$ - Non-latching
 \item PSY2A Current $PREAD \leq PLOW$ - Non-latching
 \item PSY2A Current $\mid$PREAD-PSET$\mid$  $\geq$ PDiff - Non-latching
 \item PSY2A Communication Fault - Initialize to remove latch
 \item PSX3A Current $PREAD \geq PHIGH$ - Non-latching
 \item PSX3A Current $PREAD \leq PLOW$ - Non-latching
 \item PSX3A Current $\mid$PREAD-PSET$\mid$  $\geq$ PDiff - Non-latching
 \item PSX3A Communication Fault - Initialize to remove latch
 \item PSY3A Current $PREAD \geq PHIGH$ - Non-latching
 \item PSY3A Current $PREAD \leq PLOW$ - Non-latching
 \item PSY3A Current $\mid$PREAD-PSET$\mid$  $\geq$ PDiff - Non-latching
 \item PSY3A Communication Fault - Initialize to remove latch
 \item X2A/X3A/Y2A/Y3A Watchdog - Initialize to remove latch
 \item GCC Initializing - Non-latching

\color{red}
Already checked in Particle Beam Interlock section of System Control Testing.
\color{black}

 \item GCC Off - Non-latching

\color{red}
Already checked in Particle Beam Interlock section of System Control Testing.
\color{black}

 \item GCC Shutting Down - Non-latching

\color{red}
Already checked in Particle Beam Interlock section of System Control Testing.
\color{black}

 \item GCC Current $PREAD \geq PHIGH$ - Non-latching

\color{red}
Already checked in Particle Beam Interlock section of System Control Testing.
\color{black}

 \item GCC Current $PREAD \leq PLOW$ - Non-latching

\color{red}
Already checked in Particle Beam Interlock section of System Control Testing.
\color{black}

 \item GCC Current $\mid$PREAD-PSET$\mid$  $\geq$ PDiff - Non-latching

\color{red}
Already checked in Particle Beam Interlock section of System Control Testing.
\color{black}

 \item GCC Communication Fault - Initialize to remove latch

\color{red}
Already checked in Particle Beam Interlock section of System Control Testing.
\color{black}

 \item GCC Watchdog - Initialize to remove latch

\color{red}
Already checked in Particle Beam Interlock section of System Control Testing.
\color{black}

\end{enumerate}


\subsection{Safety}

Loss of control of the beamline magnets will result in loss of control of the particle beam if one is being produced.  The primary protection against this is that the control system monitors the output current of the control magnets.  If the output current is out of tolerance or in question (e.g., see section \ref{sect:cyc-equip-ctl-beamline-sm23a-state-monitors-beam-interlocks} for a description) the control system will shut off the particle beam as described in section \ref{sect:cyc-equip-ctl-safety-sys-control-beam-control}.  The back up for this is provided by the particle beam hardwired safety interlock system (HSIS) (section \ref{sect:cyc-equip-ctl-safety-sys-hsis-beam}) that monitors the particle beam position and beam losses described in section \ref{ch:cyc-equip-ctl-beam-diagnostics}.


\subsection{Analog Control Parameters}

\begin{enumerate}
 \item PSX2A Current PSET  n.nn A
 \item PSX2A Current PLOW  n.nn A
 \item PSX2A Current PHIGH n.nn A
 \item PSX2A Current PDiff n.n A
 \item PSX2A Current PSEN  n
 \item PSY2A Current PSET  n.nn A
 \item PSY2A Current PLOW  n.nn A
 \item PSY2A Current PHIGH n.nn A
 \item PSY2A Current PDiff n.n A
 \item PSY2A Current PSEN  n
 \item PSX3A Current PSET  n.nn A
 \item PSX3A Current PLOW  n.nn A
 \item PSX3A Current PHIGH n.nn A
 \item PSX3A Current PDiff n.n A
 \item PSX3A Current PSEN  n
 \item PSY3A Current PSET  n.nn A
 \item PSY3A Current PLOW  n.nn A
 \item PSY3A Current PHIGH n.nn A
 \item PSY3A Current PDiff n.n A
 \item PSY3A Current PSEN  n
 \item GCC Current PSET  n.nn A

\color{red}
Already checked on Analog Control Parameters section of System Control Testing.
\color{black}

 \item GCC Current PLOW  n.nn A

\color{red}
Already checked on Analog Control Parameters section of System Control Testing.
\color{black}

 \item GCC Current PHIGH n.nn A

\color{red}
Already checked on Analog Control Parameters section of System Control Testing.
\color{black}

 \item GCC Current PDiff n.n A

\color{red}
Already checked on Analog Control Parameters section of System Control Testing.
\color{black}

 \item GCC Current PSEN  n

\color{red}
Already checked on Analog Control Parameters section of System Control Testing.
\color{black}

\end{enumerate}


\subsection{Parameter Limits}

\begin{enumerate}
 \item PSX2A Current MinLimit = -1.3 A
 \item PSX2A Current MaxLimit = 1.3 A
 \item PSX2A Current PLOWWarn = PLOW + 0.5 A
 \item PSX2A Current PHIGHWarn = PHIGH - 0.5 A
 \item PSY2A Current MinLimit = -1.3 A
 \item PSY2A Current MaxLimit = 1.3 A
 \item PSY2A Current PLOWWarn = PLOW + 0.5 A
 \item PSY2A Current PHIGHWarn = PHIGH - 0.5 A
 \item PSX3A Current MinLimit = -1.3 A
 \item PSX3A Current MaxLimit = 1.3 A
 \item PSX3A Current PLOWWarn = PLOW + 0.5 A
 \item PSX3A Current PHIGHWarn = PHIGH - 0.5 A
 \item PSY3A Current MinLimit = -1.3 A
 \item PSY3A Current MaxLimit = 1.3 A
 \item PSY3A Current PLOWWarn = PLOW + 0.5 A
 \item PSY3A Current PHIGHWarn = PHIGH - 0.5 A
 \item GCC Current MinLimit = -1.3 A

\color{red}
Already checked in Parameter Limits section of System Control Testing.
\color{black}

 \item GCC Current MaxLimit = 1.3 A

\color{red}
Already checked in Parameter Limits section of System Control Testing.
\color{black}

 \item GCC Current PLOWWarn = PLOW + 0.5 A

\color{red}
Already checked in Parameter Limits section of System Control Testing.
\color{black}

 \item GCC Current PHIGHWarn = PHIGH - 0.5 A

\color{red}
Already checked in Parameter Limits section of System Control Testing.
\color{black}

\end{enumerate}

\subsection{Analog Display Parameters}

\begin{enumerate}
 \item PSX2A Current PSET  n.nn A
 \item PSX2A Current PLOW  n.nn A
 \item PSX2A Current PHIGH n.nn A
 \item PSX2A Current PDiff n.nn A
 \item PSX2A Current PSEN  n
 \item PSX2A Voltage PREAD n.nn V
 \item X2A Resistance PREAD n.n $\Omega$
 \item X2A Resistance PLOW n.n $\Omega$
 \item X2A Resistance PHIGH n.n $\Omega$
 \item PSX3A Current PSET  n.nn A
 \item PSX3A Current PLOW  n.nn A
 \item PSX3A Current PHIGH n.nn A
 \item PSX3A Current PDiff n.nn A
 \item PSX3A Current PSEN  n
 \item PSX3A Voltage PREAD n.nn V
 \item X3A Resistance PREAD n.n $\Omega$
 \item X3A Resistance PLOW n.n $\Omega$
 \item X3A Resistance PHIGH n.n $\Omega$
 \item PSY2A Current PSET  n.nn A
 \item PSY2A Current PLOW  n.nn A
 \item PSY2A Current PHIGH n.nn A
 \item PSY2A Current PDiff n.nn A
 \item PSY2A Current PSEN  n
 \item PSY2A Voltage PREAD n.nn V
 \item Y2A Resistance PREAD n.n $\Omega$
 \item Y2A Resistance PLOW n.n $\Omega$
 \item Y2A Resistance PHIGH n.n $\Omega$
 \item PSY3A Current PSET  n.nn A
 \item PSY3A Current PLOW  n.nn A
 \item PSY3A Current PHIGH n.nn A
 \item PSY3A Current PDiff n.nn A
 \item PSY3A Current PSEN  n
 \item PSY3A Voltage PREAD n.nn V
 \item Y3A Resistance PREAD n.n $\Omega$
 \item Y3A Resistance PLOW n.n $\Omega$
 \item Y3A Resistance PHIGH n.n $\Omega$
 \item GCC Current PSET  n.nn A

\color{red}
Already checked on Analog Control Parameters section of System Control Testing.
\color{black}

 \item GCC Current PLOW  n.nn A

\color{red}
Already checked on Analog Control Parameters section of System Control Testing.
\color{black}

 \item GCC Current PHIGH n.nn A

\color{red}
Already checked on Analog Control Parameters section of System Control Testing.
\color{black}

 \item GCC Current PDiff n.nn A

\color{red}
Already checked on Analog Control Parameters section of System Control Testing.
\color{black}

 \item GCC Current PSEN  n

\color{red}
Already checked on Analog Control Parameters section of System Control Testing.
\color{black}

 \item GCC Voltage PREAD n.nn V

\color{red}
Not in control system.
\color{black}

 \item GCC Resistance PREAD n.n $\Omega$

\color{red}
Not in control system.
\color{black}

 \item GCC Resistance PLOW n.n $\Omega$

\color{red}
Not in control system.
\color{black}

 \item GCC Resistance PHIGH n.n $\Omega$

\color{red}
Not in control system.
\color{black}

\end{enumerate}

\subsection{Implementation Notes}

The SM23A Steering Magnets and Gantry Correction Coil power supplies are commercial Kikusui DC supplies.  Digital control and monitoring of this power supply is via the MOD 1 PLC.  Analog control and monitoring are done by the control system using by way of Ethernet/GPIB via a Agilent E5810A GPIB gateway.

\end{document}
